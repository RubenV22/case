\subsubsection{Ogenblikkelijke totale rotatiesnelheidsvector}
Om de ogenblikkelijke totale rotatiesnelheidsvector $\overrightarrow{\omega}_{w}$ te bepalen worden de drie afzonderlijke rotatiesnelheidsvectoren uitgedrukt in het wereldassenstelsel en vervolgens opgeteld. De totale ogenblikkelijke rotatievector is immers de vectori\"ele som van de afzonderlijke ogenblikkelijke rotatievectoren.\\
\begin{equation}
\begin{split}
\overrightarrow{\omega_{tot}}
&=\overrightarrow{\omega_{g}}+\overrightarrow{\omega_{i}}+\overrightarrow{\omega_{w}}
%&=R^{x'y'z' \rightarrow xyz}\,\overrightarrow{\omega}_{g}'+R^{x'y'z' \rightarrow xyz}\,\overrightarrow{\omega}_{i}'+R^{x'''y'''z''' \rightarrow xyz}\,\overrightarrow{\omega}_{w}'''
\end{split}
\end{equation}



Deze termen worden nu afzonderlijk bepaald.
%omega_g
\begin{equation*}
\begin{split}
\overrightarrow{\omega_{g}}
&=R^{x'y'z' \rightarrow xyz}\,\overrightarrow{\omega}_{g}'
&=	  \begin{bmatrix}
      1 & 0 & 0\\
      0 & \cos(\alpha) & -\sin{\alpha}\\ 
      0 & \sin{\alpha} & \cos(\alpha)\
      \end{bmatrix}
      \begin{bmatrix}
      0\\
      0\\
      \omega_{g}\
      \end{bmatrix}     
&=	  \begin{bmatrix}
      0\\
      -\sin(\alpha)\,\omega_{g}\\
      \cos(\alpha)\,\omega_{g}\
      \end{bmatrix}
\end{split}
\end{equation*}

%omega_i
\begin{equation*}
\begin{split}
\overrightarrow{\omega_{i}}
&=R^{x'y'z' \rightarrow xyz}\,\overrightarrow{\omega_{i}}'
&=	  \begin{bmatrix}
      1 & 0 & 0\\
      0 & \cos(\alpha) & -\sin{\alpha}\\ 
      0 & \sin{\alpha} & \cos(\alpha)\
      \end{bmatrix}
      \begin{bmatrix}
      0\\
      \omega_{i}\\
      0\
      \end{bmatrix}     
&=	  \begin{bmatrix}
      0\\
      \cos(\alpha)\,\omega_{i}\\
      \sin(\alpha)\,\omega_{i}\
      \end{bmatrix}
\end{split}
\end{equation*}

%omega_w
\begin{equation*}
\begin{split}
\overrightarrow{\omega_{w}}
&=R^{x'''y'''z''' \rightarrow xyz}\,\overrightarrow{\omega_{w}}'''
&=	  \begin{bmatrix}
      \cos(\beta) & 0 & \sin(\beta)\\
      \sin(\alpha)\sin(\beta) & \cos(\alpha) & -\sin(\alpha)\cos(\beta)\\
      -\cos(\alpha)\sin(\beta) & \sin(\alpha) & \cos(\alpha)\cos(\beta)\  
      \end{bmatrix}
      \begin{bmatrix}
      -\omega_{w}\\
      0\\
      0\
      \end{bmatrix}\\
&=    \begin{bmatrix}
      -\cos(\beta)\,\omega_{w}\\
      -\sin(\alpha)\sin(\beta)\,\omega_{w}\\
      \cos(\alpha)\sin(\beta)\,\omega_{w}\
      \end{bmatrix}
\end{split}
\end{equation*}

De ogenblikkelijke rotatievector wordt dus weergegeven door de volgende vergelijking.
\begin{equation*}
\overrightarrow{\omega_{tot}}=
	\begin{bmatrix}
	-\cos(\beta)\,\omega_{w}\\
	\cos(\alpha)\,\omega_{i} - \sin(\alpha)(\omega_{g}+\sin{\beta}\,\omega_{w})\\
	\sin(\alpha)\,\omega_{i} + \cos(\alpha)(\omega_{g}+\sin{\beta}\,\omega_{w})\
	\end{bmatrix}
\end{equation*}

\subsubsection{Ogenblikkelijke totale rotatieversnellingsvector}
De totale rotatieversnellingsvector is de afgeleide van de rotatiesnelheidsvector naar de tijd. Deze vector is de som van termen die de verandering in grootte van de rotatiesnelheid weergeven en van termen die de verandering in richting van de rotatiesnelheid weergeven.

\begin{equation}
\overrightarrow{\alpha_{tot}}=\frac{d\overrightarrow{\omega_{tot}}}{dt}=\overrightarrow{\alpha_{g}}+\overrightarrow{\alpha_{w}}+\overrightarrow{\alpha_{i}}+\overrightarrow{\omega_{g}}\times\overrightarrow{\omega_{i}}+\overrightarrow{\omega_{g}}\times\overrightarrow{\omega_{w}}+\overrightarrow{\omega_{i}}\times\overrightarrow{\omega_{w}}
\end{equation}

Al deze termen worden nu afzonderlijk bepaald.

%alpha_g
\begin{equation*}
\begin{split}
\overrightarrow{\alpha_{g}}&=R^{x'y'z' \rightarrow xyz}
	\begin{bmatrix}
	0\\
	0\\
	\alpha_{g}\
	\end{bmatrix}
	&=\begin{bmatrix}
	0\\
	-\sin(\alpha)\alpha_{g}\\
	\cos(\alpha)\alpha_{g}\
	\end{bmatrix}
\end{split}
\end{equation*}

%alpha_i
\begin{equation*}
\begin{split}
\overrightarrow{\alpha_{i}}&=R^{x'y'z' \rightarrow xyz}
	\begin{bmatrix}
	0\\
	\alpha_{i}\\
	0\
	\end{bmatrix}
	&=\begin{bmatrix}
	0\\
	\cos(\alpha)\alpha_{i}\\
	\sin(\alpha)\alpha_{i}\
	\end{bmatrix}
\end{split}
\end{equation*}

%omega_w
\begin{equation*}
\begin{split}
\overrightarrow{\alpha_{w}}&=R^{x'''y'''z''' \rightarrow xyz}
	\begin{bmatrix}
	\alpha_{w}\\
	0\\
	0\
	\end{bmatrix}
	&=\begin{bmatrix}
	\cos(\beta)\alpha_{w}\\
	\sin(\alpha)\sin(\beta)\alpha_{w}\\
	-\cos(\alpha)\sin(\beta)\alpha_{w}\
	\end{bmatrix}
\end{split}
\end{equation*}

%kruis omega_g - omega_i
\begin{equation*}
\begin{split}
\overrightarrow{\omega_{g}}\times\overrightarrow{\omega_{i}}
&=	\begin{vmatrix}
	\overrightarrow{e_{x}} & \overrightarrow{e_{y}} & \overrightarrow{e_{z}}\\
	\omega_{gx} & \omega_{gy} & \omega_{gz}\\
	\omega_{ix} & \omega_{iy} & \omega_{iz}\
	\end{vmatrix}
&=	\begin{bmatrix}
	-\omega_{i}\omega_{g}\\
	0\\
	0\
	\end{bmatrix}
\end{split}
\end{equation*}

%kruis omega_g - omega_w
\begin{equation*}
\begin{split}
\overrightarrow{\omega_{g}}\times\overrightarrow{\omega_{w}}
&=	\begin{vmatrix}
	\overrightarrow{e_{x}} & \overrightarrow{e_{y}} & \overrightarrow{e_{z}}\\
	\omega_{gx} & \omega_{gy} & \omega_{gz}\\
	\omega_{wx} & \omega_{wy} & \omega_{wz}\
	\end{vmatrix}
&=	\begin{bmatrix}
	0\\
	-\cos(\alpha)\cos(\beta)\omega_{g}\omega_{w}\\
	-\sin(\alpha)\cos(\beta)\omega_{g}\omega_{w}\
	\end{bmatrix}
\end{split}
\end{equation*}

%kruis omega_i - omega_w
\begin{equation*}
\begin{split}
\overrightarrow{\omega_{i}}\times\overrightarrow{\omega_{w}}
&=	\begin{vmatrix}
	\overrightarrow{e_{x}} & \overrightarrow{e_{y}} & \overrightarrow{e_{z}}\\
	\omega_{ix} & \omega_{iy} & \omega_{iz}\\
	\omega_{wx} & \omega_{wy} & \omega_{wz}\
	\end{vmatrix}
&=	\begin{bmatrix}
	\sin(\beta)\omega_{i}\omega_{w}\\
	-\sin(\alpha)\cos(\beta)\omega_{i}\omega_{w}\\
	\cos(\beta)\cos(\beta)\omega_{i}\omega_{w}\
	\end{bmatrix}
\end{split}
\end{equation*}

De totale ogenblikkelijke rotatieversnellingsvector wordt dus door de volgende formule gegeven.

\begin{equation*}
\begin{split}
\overrightarrow{\alpha_{tot}}&=\frac{d\overrightarrow{\omega_{tot}}}{dt}\\
&=	\begin{bmatrix}
	\cos(\beta)\alpha_{w}-\omega_{g}\omega_{i}+\sin(\beta)\omega_{i}\omega_{w}\\
	-\sin(\alpha)\alpha_{g}+\cos(\alpha)\alpha_{i}+\sin(\alpha)\sin(\beta)\alpha_{w}-\cos(\alpha)\cos(\beta)\omega_{g}\omega_{w}-\sin(\alpha)\sin(\beta)\omega_{i}\omega_{w}\\
	\cos(\alpha)\alpha_{g}+\sin(\alpha)\alpha_{i}-\cos(\alpha)\sin(\beta)\alpha_{w}-\sin(\alpha)\cos(\beta)\omega_{g}\omega_{w}+\cos(\alpha)\cos(\beta)\omega_{i}\omega_{w}\
	\end{bmatrix}
\end{split}
\end{equation*}