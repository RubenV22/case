%Inleidende tekst
Om de ogenblikkelijke totale rotatiesnelheidsvector $\overrightarrow{\omega}_{w}$ te bepalen worden de drie afzonderlijke rotatiesnelheidsvectoren uitgedrukt in het wereldassenstelsel en vervolgens opgeteld.
\begin{equation*}
\overrightarrow{\omega}_{w}=R^{x'y'z' \rightarrow xyz}\,\overrightarrow{\omega}_{g}'+R^{x'y'z' \rightarrow xyz}\,\overrightarrow{\omega}_{i}'+R^{x'''y'''z''' \rightarrow xyz}\,\overrightarrow{\omega}_{w}'''
\end{equation*}



Deze termen worden nu afzonderlijk bepaald.
%omega_g
\begin{equation*}
\begin{split}
R^{x'y'z' \rightarrow xyz}\,\overrightarrow{\omega}_{g}'
&=	  \begin{bmatrix}
      1 & 0 & 0\\
      0 & \cos(\alpha) & -\sin{\alpha}\\ 
      0 & \sin{\alpha} & \cos(\alpha)\
      \end{bmatrix}
      \begin{bmatrix}
      0\\
      0\\
      \omega_{g}\
      \end{bmatrix}     
&=	  \begin{bmatrix}
      0\\
      -\sin{\alpha}\,\omega_{g}\\
      \cos(\alpha)\,\omega_{g}\
      \end{bmatrix}
\end{split}
\end{equation*}

%omega_i
\begin{equation*}
\begin{split}
R^{x'y'z' \rightarrow xyz}\,\overrightarrow{\omega}_{g}'
&=	  \begin{bmatrix}
      1 & 0 & 0\\
      0 & \cos(\alpha) & -\sin{\alpha}\\ 
      0 & \sin{\alpha} & \cos(\alpha)\
      \end{bmatrix}
      \begin{bmatrix}
      0\\
      \omega_{i}\\
      0\
      \end{bmatrix}     
&=	  \begin{bmatrix}
      0\\
      \cos{\alpha}\,\omega_{i}\\
      \sin(\alpha)\,\omega_{i}\
      \end{bmatrix}
\end{split}
\end{equation*}

%omega_w
\begin{equation*}
\begin{split}
R^{x'''y'''z''' \rightarrow xyz}\,\overrightarrow{\omega}_{w}'''
&=	  \begin{bmatrix}
      \cos(\beta) & 0 & \sin(\beta)\\
      \sin(\alpha)\sin(\beta) & \cos(\alpha) & -\sin(\alpha)\cos(\beta)\\
      -\cos(\alpha)\sin(\beta) & \sin(\alpha) & \cos(\alpha)\cos(\beta)\  
      \end{bmatrix}
      \begin{bmatrix}
      -\omega_{w}\\
      0\\
      0\
      \end{bmatrix}
&=    \begin{bmatrix}
      -\cos(\beta)\,\omega_{w}\\
      -\sin(\alpha)\sin(\beta)\,\omega_{w}\\
      \cos(\alpha)\sin(\beta)\,\omega_{w}\
      \end{bmatrix}
\end{split}
\end{equation*}

De ogenblikkelijke rotatievector wordt dus weergegeven door door de volgende vergelijking.
\begin{equation*}
\overrightarrow(\omega)_{w}=
\begin{bmatrix}
	-\cos(\beta)\,\omega_{w}\\
	\cos(\alpha)\,\omega_{i} - \sin(\alpha)(\omega_{g}+\sin{\beta}\,\omega_{w})\\
	\sin(\alpha)\,\omega_{i} + \cos(\alpha)(\omega_{g}+\sin{\beta}\,\omega_{w})\
\end{bmatrix}
\end{equation*}