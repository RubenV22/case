\subsubsection{Ogenblikkelijke snelheid}
Voor het berekenen van de snelheid van D wordt een gelijkaardige methode gebruikt als voor de snelheid van C. Aangezien D zich niet op het wiel bevindt, heeft de rotatie van het wiel nog steeds geen invloed op de snelheid.

\begin{equation*}
\overrightarrow{v_{D}}=\overrightarrow{v_{A}}+\overrightarrow{\omega_{g}}\times(\overrightarrow{r_{D}}-\overrightarrow{r_{A}})+\overrightarrow{\omega_{i}}\times(\overrightarrow{r_{D}}-\overrightarrow{r_{B}})
\end{equation*}

%rel plaats D-A
\begin{equation*}
\begin{split}
\overrightarrow{r_{D}}-\overrightarrow{r_{A}}
&=	(\overrightarrow{r_{D}}-\overrightarrow{r_{C}}) + 		(\overrightarrow{r_{C}}-\overrightarrow{r_{A}})\\
&=	R^{x'''y'''z''' \rightarrow xyz}
	\begin{bmatrix}
	\frac{3}{4}l_{4}\\
	0\\
	\frac{1}{4}l_{3}\
	\end{bmatrix}
	+\begin{bmatrix}
	r_{CAx}\\
	r_{CAy}\\
	r_{CAz}\
	\end{bmatrix}\\
&=	\begin{bmatrix}
	-1/4\,l_{4}\,\cos \left( \beta \right) -3/4\,l_{3}\,\sin \left( \beta \right) +l_{1}\\
	3/4\,\sin \left( \beta \right) \sin \left( \alpha \right) l_{4}-1/4\,\cos\left( \beta \right) \sin \left( \alpha \right) l_{3}-\sin \left( \alpha \right)  \left( -l_{3}\,\cos \left( \beta \right) +l_{4}\,\sin\left( \beta \right) +l_{2} \right) \\
	-3/4\,\sin\left( \beta \right) \cos \left( \alpha \right) l_{4}+1/4\,\cos\left( \beta \right) \cos \left( \alpha \right) l_{3}+\cos \left( \alpha \right)  \left( -l_{3}\,\cos \left( \beta \right) +l_{4}\,\sin\left( \beta \right) +l_{2} \right) \
	\end{bmatrix}
\end{split}
\end{equation*}

%kruis omega_g - rel plaats D-A
\begin{equation*}
\begin{split}
\overrightarrow{\omega_{g}}\times(\overrightarrow{r_{D}}-\overrightarrow{r_{A}})
&=	\begin{vmatrix}
	\overrightarrow{e_{x}} & \overrightarrow{e_{y}} & \overrightarrow{e_{z}}\\
	\omega_{gx} & \omega_{gy} & \omega_{gz}\\
	v_{DAx} & v_{DAy} & v_{DAz}\
	\end{vmatrix}
&=	\begin{bmatrix}
	0\\
	-1/4\,\omega_{g}\,\cos\left( \alpha \right)  \left( 3\,l_{3}\,\sin \left( \beta \right) +l_{4}\,\cos \left( \beta \right) -4\,l_{1} \right) \\
	-1/4\,\omega_{g}\,\sin \left( \alpha \right)  \left( 3\,l_{3}\,\sin\left( \beta \right) +l_{4}\,\cos \left( \beta \right) -4\,l_{1}\right) \
	\end{bmatrix}
\end{split}
\end{equation*}

%rel plaats D-B
\begin{equation*}
\begin{split}
\overrightarrow{r_{D}}-\overrightarrow{r_{B}}
&=	(\overrightarrow{r_{D}}-\overrightarrow{r_{C}}) + 		(\overrightarrow{r_{C}}-\overrightarrow{r_{B}})\\
&=	\begin{bmatrix}
	-1/4\,l_{4}\,\cos \left( \beta \right) -3/4\,l_{3}\,\sin \left( \beta \right) \\
	3/4\,\sin\left( \beta \right) \sin \left( \alpha \right) l_{4}-1/4\,\cos\left( \beta \right) \sin \left( \alpha \right) l_{3}-\sin \left( \alpha \right)  \left( -l_{3}\,\cos \left( \beta \right) +l_{4}\,\sin\left( \beta \right)  \right) \\
	-3/4\,\sin \left( \beta \right) \cos \left( \alpha \right) l_{4}+1/4\,\cos \left( \beta\right) \cos \left( \alpha \right) l_{3}+\cos \left( \alpha \right) \left( -l_{3}\,\cos \left( \beta \right) +l_{4}\,\sin \left( \beta\right)  \right) \	
	\end{bmatrix}
\end{split}
\end{equation*}

%kruis omega_i - rel plaats D-B
\begin{equation*}
\begin{split}
\overrightarrow{\omega_{i}}\times(\overrightarrow{r_{D}}-\overrightarrow{r_{B}})
&=	\begin{vmatrix}
	\overrightarrow{e_{x}} & \overrightarrow{e_{y}} & \overrightarrow{e_{z}}\\
	\omega_{ix} & \omega_{iy} & \omega_{iz}\\
	v_{DBx} & v_{DBy} & v_{DBz}\
	\end{vmatrix}
&=	\begin{bmatrix}
	-1/4\,\omega_{i}\, \left( 3\,l_{3}\,\cos\left( \beta \right) -l_{4}\,\sin \left( \beta \right)  \right) \\
	1/4\,\omega_{i}\,\sin \left( \alpha \right) \left( 3\,l_{3}\,\sin \left( \beta \right) +l_{4}\,\cos \left( \beta\right)  \right) \\
	1/4\,\omega_{i}\,\cos \left( \alpha \right)  \left( 3\,l_{3}\,\sin \left( \beta \right) +l_{4}\,\cos \left( \beta \right)  \right) \
	\end{bmatrix}
\end{split}
\end{equation*}

De totale snelheid van D wordt dus door de volgende gelijkheid gegeven.
\begin{equation*}
\begin{split}
\overrightarrow{a_{D}}
&=	\begin{bmatrix}
	-1/4\,\omega_{i}\, \left( 3\,l_{3}\,\cos\left( \beta \right) -l_{4}\,\sin \left( \beta \right)  \right) \\
	\cos \left( \alpha \right) v_{v}-1/4\,\omega_{g}\,\cos \left( \alpha \right)  \left( 3\,l_{3}\,\sin \left( \beta\right) +l_{4}\,\cos \left( \beta \right) -4\,l_{1} \right) -1/4\,\omega_{i}\,\sin \left( \alpha \right)  \left( 3\,l_{3}\,\sin \left( \beta \right) +l_{4}\,\cos \left( \beta \right)  \right) \\
	\sin \left( \alpha \right) v_{v}-1/4\,\omega_{g}\,\sin \left( \alpha \right)  \left( 3\,l_{3}\,\sin \left( \beta\right) +l_{4}\,\cos \left( \beta \right) -4\,l_{1} \right) +1/4\,\omega_{i}\,\cos \left( \alpha \right)  \left( 3\,l_{3}\,\sin \left( \beta \right) +l_{4}\,\cos \left( \beta \right)  \right) \
	\end{bmatrix}
\end{split}
\end{equation*}

\subsubsection{Ogenblikkelijke versnelling}
Ook voor de ogenblikkelijke versnelling van D wordt een gelijkaardige methode gebruikt als voor C.

\begin{equation*}
\begin{split}
\overrightarrow{a_{D}}&=\frac{d\overrightarrow{v_{D}}}{dt}
&=	\overrightarrow{a_{A}} + \frac{d\overrightarrow{\omega_{g}}}{dt}\times(\overrightarrow{r_{D}}-\overrightarrow{r_{A}}) + \overrightarrow{\omega_{g}}\times(\overrightarrow{v_{D}}-\overrightarrow{v_{A}}) + \frac{d\overrightarrow{\omega_i}}{dt}\times(\overrightarrow{r_{D}}-\overrightarrow{r_{B}})+\overrightarrow{\omega_{i}}\times(\overrightarrow{v_{D}}-\overrightarrow{v_{B}})
\end{split}
\end{equation*}

Opnieuw worden al deze termen afzonderlijk uitgerekend.
