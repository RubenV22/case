\subsubsection{Ogenblikkelijke impulsvector landingsgestel}
De impuls van het landingsgestel is gemakkelijk te berekenen indien de snelheidsvector van het massacentrum en de massa van het landingsgestel bepaald zijn. Deze moeten dan alleen nog met elkaar vermenigvuldigd worden.
\begin{equation}
\begin{split}
\overrightarrow{{p}_{l}}
&=m_{l}\,\overrightarrow{{v}_{D}}\\
%&=	  \begin{bmatrix}
%1/4\,m_{l}\,\omega_{i}\, \left( l_{4}\,\sin\left( \beta \right) -3\,l_{3}\,\cos \left( \beta \right)  \right)\\
%%
%m_{l}\, \left( \cos \left( \alpha \right) v_{v}-1
%/4\,\omega_{g}\,\cos \left( \alpha \right)  \left( 3\,l_{3}\,\sin
% \left( \beta \right) +l_{4}\,\cos \left( \beta \right) -4\,l_{1}
% \right) -1/4\,\omega_{i}\,\sin \left( \alpha \right)  \left( 3\,l_{3}
%\,\sin \left( \beta \right) +l_{4}\,\cos \left( \beta \right) 
% \right)  \right) \\
%%
%m_{l}\, \left( \sin \left(\alpha \right) v_{v}-1/4\,\omega_{g}\,\sin \left( \alpha \right) \left( 3\,l_{3}\,\sin \left( \beta \right) +l_{4}\,\cos \left( \beta\right) -4\,l_{1} \right) +1/4\,\omega_{i}\,\cos \left( \alpha\right)  \left( 3\,l_{3}\,\sin \left( \beta \right) +l_{4}\,\cos\left( \beta \right)  \right)  \right) \
%\end{bmatrix}
\end{split}
\end{equation}

\subsubsection{Verandering impulsvector landingsgestel}
De verandering van de impulsvector is gelijk aan de afgeleide van de impulsvector. Aangezien de massa van het landingsgestel niet verandert, moet alleen de snelheid in de formule uit het vorige onderdeel vervangen worden door de versnelling van het massacentrum van het landingsgestel.
\begin{equation}
\begin{split}
\frac{d\overrightarrow{{p}_{l}}}{dt}
&=m_{l}\,\overrightarrow{{a}_{D}}\\
%&=	  \begin{bmatrix}
%m_{l}\, \left( 1/4\,{\omega_{g}}^{2} \left( 3\,l_{3}\,\sin \left( \beta \right) +l_{4}\,\cos \left( \beta \right) -4\,l_{1} \right) +1/4\,\alpha_{i}\, \left( l_{4}\,\sin \left( \beta\right) -3\,l_{3}\,\cos \left( \beta \right)  \right) +1/4\,{\omega_{i}}^{2} \left( 3\,l_{3}\,\sin \left( \beta \right) +l_{4}\,\cos\left( \beta \right)  \right)  \right) \\
%%
%m_{l}\,\left( {a_{v}}\,\cos \left( \alpha \right) -1/4\,\alpha_{g}\,\cos\left( \alpha \right)  \left( 3\,l_{3}\,\sin \left( \beta \right) +l_{4}\,\cos \left( \beta \right) -4\,l_{1} \right) +1/4\,\cos \left( \alpha \right) \omega_{g}\,\omega_{i}\, \left( l_{4}\,\sin \left( \beta \right) -3\,l_{3}\,\cos \left( \beta \right)  \right) +1/4\,\sin\left( \beta \right) \cos \left( \alpha \right) l_{4}\,\omega_{g}\,\omega_{i}-3/4\,\cos \left( \beta \right) \cos \left( \alpha \right) l_{3}\,\omega_{g}\,\omega_{i}-3/4\,\sin \left( \beta \right) \sin\left( \alpha \right) l_{3}\,\alpha_{i}-1/4\,\cos \left( \beta\right) \sin \left( \alpha \right) l_{4}\,\alpha_{i}+1/4\,\sin\left( \alpha \right) {\omega_{i}}^{2} \left( l_{4}\,\sin \left( \beta \right) -3\,l_{3}\,\cos \left( \beta \right)  \right)  \right) \\
%%
%m_{l}\, \left( {a_{v}}\,\sin \left( \alpha\right) -1/4\,\alpha_{g}\,\sin \left( \alpha \right)  \left( 3\,l_{3}\,\sin \left( \beta \right) +l_{4}\,\cos \left( \beta \right) -4\,l_{1} \right) +1/4\,\sin \left( \alpha \right) \omega_{g}\,\omega_{i}\,\left( l_{4}\,\sin \left( \beta \right) -3\,l_{3}\,\cos \left( \beta\right)  \right) +1/4\,\sin \left( \beta \right) \sin \left( \alpha\right) l_{4}\,\omega_{g}\,\omega_{i}-3/4\,\cos \left( \beta \right) \sin \left( \alpha \right) l_{3}\,\omega_{g}\,\omega_{i}+3/4\,\sin\left( \beta \right) \cos \left( \alpha \right) l_{3}\,\alpha_{i}+1/4\,\cos \left( \beta \right) \cos \left( \alpha \right) l_{4}\,\alpha_{i}-1/4\,\cos \left( \alpha \right) {\omega_{i}}^{2} \left( l_{4}\,\sin\left( \beta \right) -3\,l_{3}\,\cos \left( \beta \right)  \right) \right) \
%\end{bmatrix}
\end{split}
\end{equation}

\subsubsection{Ogenblikkelijke impulsvector wiel}
Voor het wiel volgt een zeer gelijkaardige redenering als voor het landingsgestel. De massa van het voorwerp wordt vermenigvuldigd met de snelheidsvector van het massacentrum om de impuls te bekomen. 
\begin{equation}
\begin{split}
\overrightarrow{{p}_{w}}
&=m_{w}\,\overrightarrow{{v}_{C}}
%&=	  \begin{bmatrix}
%m_{w}\,\omega_{i}\, \left( -l_{3}\,\cos\left( \beta \right) +l_{4}\,\sin \left( \beta \right)  \right) \\
%%
%m_{w}\, \left( \cos \left( \alpha \right) v_{v}-\omega_{g}\,\cos \left( \alpha \right)  \left( l_{3}\,\sin \left( \beta \right) +l_{4}\,\cos \left( \beta \right) -l_{1} \right) -\omega_{i}\,\sin \left( \alpha \right)  \left( l_{3}\,\sin \left( \beta\right) +l_{4}\,\cos \left( \beta \right)  \right)  \right) \\
%%
%m_{w}\, \left( \sin \left( \alpha \right) v_{v}-\omega_{g}\,\sin \left( \alpha \right)  \left( l_{3}\,\sin \left( \beta \right) +l_{4}\,\cos \left( \beta \right) -l_{1} \right) +\omega_{i}\,\cos \left( \alpha \right)  \left( l_{3}\,\sin \left( \beta\right) +l_{4}\,\cos \left( \beta \right)  \right)  \right) \
%      \end{bmatrix}
\end{split}
\end{equation}

\subsubsection{Verandering impulsvector wiel}
De afgeleide van de impulsvector is ook hier gelijk aan het product van de massa van het wiel met de versnellingsvector van het massacentrum van het wiel.
\begin{equation}
\begin{split}
\frac{d\overrightarrow{{p}_{w}}}{dt}
&=m_{w}\,\overrightarrow{{a}_{C}}
%&=	  \begin{bmatrix}
%m_{w}\, \left( {\omega_{g}}^{2} \left( l_{3}\,\sin \left( \beta \right) +l_{4}\,\cos \left( \beta \right) -l_{1}\right) +\alpha_{i}\, \left( -l_{3}\,\cos \left( \beta \right) +l_{4}\,\sin \left( \beta \right)  \right) +{\omega_{i}}^{2} \left( l_{3}\,\sin \left( \beta \right) +l_{4}\,\cos \left( \beta \right)  \right) \right) \\
%%
%m_{w}\, \left( {a_{v}}\,\cos \left( \alpha \right) +\cos \left( \alpha \right) \alpha_{g}\, \left( -l_{3}\,\sin \left( \beta \right) -l_{4}\,\cos \left( \beta \right) +l_{1}\right) +\omega_{g}\,\omega_{i}\,\cos \left( \alpha \right)  \left( -l_{3}\,\cos \left( \beta \right) +l_{4}\,\sin \left( \beta \right) \right) +\sin \left( \beta \right) \cos \left( \alpha \right) l_{4}\,\omega_{g}\,\omega_{i}-\cos \left( \beta \right) \cos \left( \alpha\right) l_{3}\,\omega_{g}\,\omega_{i}-\sin \left( \beta \right) \sin\left( \alpha \right) l_{3}\,\alpha_{i}-\cos \left( \beta \right) \sin \left( \alpha \right) l_{4}\,\alpha_{i}+\sin \left( \alpha\right) {\omega_{i}}^{2} \left( -l_{3}\,\cos \left( \beta \right) +l_{4}\,\sin \left( \beta \right)  \right)  \right) \\
%%
%m_{w}\, \left( {\it a\_v}\,\sin \left( \alpha \right) +\sin \left( \alpha \right) \alpha_{g}\, \left( -l_{3}\,\sin \left( \beta \right) -l_{4}\,\cos \left( \beta \right) +l_{1} \right) +\omega_{g}\,\omega_{i}\,\sin \left( \alpha \right)  \left( -l_{3}\,\cos \left( \beta\right) +l_{4}\,\sin \left( \beta \right)  \right) +\sin \left( \beta\right) \sin \left( \alpha \right) l_{4}\,\omega_{g}\,\omega_{i}-\cos\left( \beta \right) \sin \left( \alpha \right) l_{3}\,\omega_{g}\,\omega_{i}+\sin \left( \beta \right) \cos \left( \alpha \right) l_{3}\,\alpha_{i}+\cos \left( \beta \right) \cos \left( \alpha \right) l_{4}\,\alpha_{i}-\cos \left( \alpha \right) {\omega_{i}}^{2} \left( -l_{3}\,\cos\left( \beta \right) +l_{4}\,\sin \left( \beta \right) \right)  \right) \
%      \end{bmatrix}
\end{split}
\end{equation}