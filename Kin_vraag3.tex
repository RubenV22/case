Om de bijdrage van de corriolisversnelling te bepalen kan de volgende formule gebruikt worden.
\begin{equation*}
\overrightarrow{a_{corriolis}}=-2(\overrightarrow{\omega}\times\overrightarrow{v_{rel}})
\end{equation*}

Om de versnelling van het voorwerp te beschrijven, wordt $\omega$ gelijkgesteld aan $\omega_{i}$. De relatieve snelheid in dit assenstelsel is de rotatiesnelheid van C rond B.

\begin{equation*}
\begin{split}
\overrightarrow{\omega}=\overrightarrow{\omega_{i}}
\end{split}
\end{equation*}

\begin{equation*}
\begin{split}
\overrightarrow{v_{rel}}&= \overrightarrow{\omega_{i}}\times(\overrightarrow{r_{C}}-\overrightarrow{r_{B}})
&=	\begin{bmatrix}
	-\omega_{i}\, \left( l_{3}\,\cos \left( \beta \right) -l_{4}\,\sin \left( \beta \right)  \right) \\
	-\omega_{i}\,\sin \left( \alpha \right)  \left( l_{3}\,\sin \left( \beta \right) +l_{4}\,\cos \left( \beta \right) \right) \\
	\omega_{i}\,\cos \left( \alpha \right) \left( l_{3}\,\sin \left( \beta \right) +l_{4}\,\cos \left( \beta\right)  \right) \
\end{bmatrix}
\end{split}
\end{equation*}

\begin{equation*}
\begin{split}
\overrightarrow{a_{corriolis}}&=
\begin{bmatrix}
0\\
2\,\cos \left( \alpha\right) \omega_{g}\,\omega_{i}\, \left( l_{3}\,\cos \left( \beta\right) -l_{4}\,\sin \left( \beta \right)  \right) \\
2\,\sin \left( \alpha \right) \omega_{g}\,\omega_{i}\, \left( l_{3}\,\cos \left( \beta \right) -l_{4}\,\sin \left( \beta \right)  \right) \
\end{bmatrix}
\end{split}
\end{equation*}